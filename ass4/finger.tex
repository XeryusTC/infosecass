There are 7 billion ($7 \cdot 10^9$) people on the planet. If all of them would create a document and sign it using both SHA1 and Tiger then they would generate $7 \cdot 10^9$ hashes of both algorithms per day, or about $2.5 \cdot 10^{12}$ hashes per year.

SHA1 uses 160 bit hashes, so there is a pool of $2^{160} \approx 10^{53}$ hashes. To find a collision we can use the method used in a birthday attack. This means that we only need to compute (well, have everyone on the planet work for us) the square root of $10^{53}$ hashes, which is about $10^{27}$. At a rate of $2.5 \cdot 10^{12}$ hashes per year this would take about $\displaystyle \frac{10^{27}}{2.5 \cdot 10^{12}} = 4 \cdot 10^{14}$.

Tiger results in 192 bit hash values, making the entire pool of hashes $2^{192} \approx 10^{64}$. Using the birthday problem we would only need to generate $10^{32}$ hashes. At $2.5 \cdot 10^{12}$ hashes per year it is likely that a collision would occur after $\displaystyle \frac{10^{32}}{2.5 \cdot 10^{12}} = 4 \cdot 10^{19}$ years. Considering that the death of the sun is predicted to be in $5 \cdot 10^{9}$ years it is very unlikely that humanity would find a collision for either algorithm using this method.

A PGP fingerprint is 20 bytes large. This means that there are $2^{160} \approx 10^{53}$ possible fingerprints out there. Since there are $7 \cdot 10^9$ people on the planet. The probability of these people getting the same fingerprint is $\displaystyle 1 - \prod\limits_{10^{53}-10^{10}}^{10^{53}} \frac{n}{10^{53}} \approx 1 - 1 = 0$. So it is safe to say that everyone will have a unique fingerprint.
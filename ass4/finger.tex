There are 7 billion ($7 \cdot 10^9$) people on the planet. If all of them would create a document and sign it using both SHA1 and Tiger then they would generate $7 \cdot 10^9$ hashes of both algorithms per day, or about $2.5 \cdot 10^{12}$ hashes per year.

SHA1 uses 160 bit hashes, so there is a pool of $2^{160} \approx 10^{53}$ hashes. When half of the possible hashes have been generated there is a reasonable chance of a hash collision. So humanity would have to generate $10^{52}$ documents to have a reasonable chance to get a hash collision. At a rate of $2.5 \cdot 10^{12}$ hashes per year this would take about $\displaystyle \frac{10^{53}}{2.5 \cdot 10^{12}} = 4 \cdot 10^{39}$ years. Considering that the death of the sun is predicted to be in $5 \cdot 10^{9}$ years it is very unlikely that humanity would achieve this feat.

Tiger results in 192 bit hash values, making the entire pool $2^{192} \approx 10^{64}$ hashes. At $2.5 \cdot 10^{12}$ hashes per year a guaranteed collision would occur after $\displaystyle \frac{10^{64}}{2.5 \cdot 10^{12}} = 4 \cdot 10^{51}$ years. But it is very likely that a collision would already occur after $4 \cdot 10^{50}$ years.
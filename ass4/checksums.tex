The problem here is that the checksums can be found on the same page as the file that is being downloaded. The idea is that if someone, say Alice, wants to  download a program, for example OpenTTD, she can check if the file she downloaded is really OpenTTD instead of a possibly malicious file placed there by a hacker.

But if the hacker can replace the download link to that file, she can also change other aspects of the served webpage, like the checksums. So although it seems that the checksums allow you to check if you downloaded the real program, this check can easily be circumvented by hackers by simply changing the checksums. 

Instead of having the checksums at the same site as the download links for the program, the checksums should be served in another way. This could for example be done by an automatic email responder that serves the checksums when asked for them. 
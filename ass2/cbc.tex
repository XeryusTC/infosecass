% !TEX root =  main.tex
The first thing you need to do to is compute an initialisation vector. This vector can be random, in which case it has to be sent with the message or a nonce can be used. The IV should not be fixed.

Now we can read in the data and encrypt it. The pseudo code for this can be found in \autoref{code:encbc}. First we read in the next plain text block. This plain text block is then \lstinline{xor}ed with the last plain text block, or if it is the first block, the IV. 

This \lstinline{xor}ed block is then encrypted using the encrypt, which results in the ciphertext. The plaintext now becomes the old plain text and we start again.

\begin{lstlisting}[caption={Encryption with CBC}, label=code:encbc]
oldPlain = initialisationVector()
while (input) {
	plain = read()
	cipher = E(key, xor(plain, oldPlain))
	oldPlain = plain
	write(cipher)
}
\end{lstlisting}

The pseudocode for decryption can be found in \autoref{code:decbc}. First, we need our IV again. Once that is calculated, we can read in the first block of ciphertext. Now we have to decrypt this block. This decrypted block gets \lstinline{xor}ed with the previous plain text. Now we got our original plain text back, so we can make that the previous plain block and write away. 

\begin{lstlisting}[caption={Decryption with CBC}, label=code:decbc]
oldPlain = initialisationVector()
while (input) {
	cipher = read()
	plain = xor(D(key, cipher), oldPlain)
	oldPlain = plain
	write(plain)
}
\end{lstlisting}